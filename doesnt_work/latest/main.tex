\documentclass[12pt]{article}
\usepackage{parskip}
\usepackage{verbatim}
\usepackage{listings}
\usepackage{epsfig}
\usepackage{tabularx}
\usepackage{graphicx}
\usepackage{caption}
\usepackage{subcaption}
\usepackage[margin=3cm]{geometry}

\title{Monitoring the Galaxy Toolbox \\ Final Report \\ COSC480}
\author{Edward Hills \\ \\ Supervisor: Dr.\ David Eyers}

\begin{document}
\vspace{-1cm}

\maketitle

\section{Assignment 1}

To be honest, I still never fully understood, how the fake reading/writing was suppose to go. I realised that we should not write actual bytes (although I do memset a couple of buffers, just for fun), but I am not sure on how awe were suppose to implement the read-ahead, write-behind buffers.

Here is a brief overview of my system.

\section{Assignment 2}

Making it message passing could help make asynchronisation easier. becasue can just og off and then be caled with data. would need to store a window of replies. in fact. it would be almost identical to the tcp sliding window protocol.

\end{document}
